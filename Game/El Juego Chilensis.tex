\documentclass{article}

\usepackage{amsmath}    % need for subequations
\usepackage{graphicx}   % need for figures
\usepackage{verbatim}   % useful for program listings
\usepackage{color}      % use if color is used in text
\usepackage{subfigure}  % use for side-by-side figures
\usepackage{hyperref}   % use for hypertext links, including those to external documents and URLs
\usepackage[spanish]{babel}
\usepackage{cancel}
\usepackage[utf8]{inputenc}
\usepackage{fullpage}
\usepackage[section]{placeins} %esto evita que imagenes se salgan de su seccion
\usepackage{cite} %referencias
\usepackage{hyperref}
\usepackage{amsfonts}
\usepackage{listings}
%\usepackage{epstopdf}
% define tabular espaciado

\newenvironment{mitab}[2]{\vspace{#1}\begin{tabular}{#2}}%
		{\end{tabular}\vspace{5mm}}
% define mi tipo de frames
\newenvironment{myFrame}[1]{
\begin{frame}[allowframebreaks]
\frametitle{#1}
\justifying
}
{\end{frame}}

%\newenvironment{myEqn}{\begin{subequations}\begin{gather}}
%{\end{gather}\end{subequations}}
% define listas

\newcommand{\be}{\begin{enumerate}}
\newcommand{\ee}{\end{enumerate}}
\newcommand{\bi}{\begin{itemize}}
\newcommand{\ei}{\end{itemize}}
\newcommand{\bc}{\begin{center}}
\newcommand{\ec}{\end{center}}

% puntaje pruebas
\newcommand{\pts[1]}{{\color{red} $\to$ \boxed{\textbf{#1 pts.}}}}

% cosas matematicas

\renewcommand{\d}{\mathsf{d}}
\newenvironment{eqn}{\begin{eqnarray*}}{\end{eqnarray*}}

\newcommand{\re}[1]{\mbox{Re\{$#1$\}}}		% la parte real
\newcommand{\im}[1]{\mbox{Im\{$#1$\}}}		% la parte imaginaria
\newcommand{\gauss}{\operatorname{Gauss}}		% gauss
\newcommand{\rect}{\sqcap}			% rect
\newcommand{\pillbox}{\operatorname{circ}}		% circ
\newcommand{\sinc}{\operatorname{sinc}}		% sinc
\newcommand{\jinc}{\operatorname{jinc}}		% jinc
\newcommand{\asinc}{\operatorname{asinc}}		% asinc
\newcommand{\sen}{\operatorname{sen}}		% seno
\newcommand{\senh}{\operatorname{senh}}		% seno
\newcommand{\triang}{\wedge}			% triangulo
\newcommand{\allint}{\ds{\int_{-\infty}^\infty}}	% integral de -infinito a +infinito
\newcommand{\midint}{\int_{0}^\infty}		% integral de cero a +infinito
\newcommand{\allsum}[1]{\sum_{#1=-\infty}^\infty} % idem suma
\newcommand{\hor}{\operatorname{\uparrow\uparrow}}		% horquilla
\newcommand{\ahor}{\operatorname{\uparrow\downarrow}}		% anti horquilla
\newcommand{\sgn}{\operatorname{sgn}}			% signo
\newcommand{\ddx}{\frac{d}{dx}}			% d/dx
\newcommand{\ddt}{\frac{d}{dt}}			% d/dt
\newcommand{\partiald}[2]{\frac{\partial#1}{\partial#2}}		% d/dx (parcial)
\newcommand{\der}[1]{#1^{\prime}}		% ' derivada
\newcommand{\derr}[1]{#1^{\prime\prime}}	% '' doble derivada
\newcommand{\FT}[1]{{\cal F}\left\{#1\right\}}		% FT
\newcommand{\IFT}[1]{{\cal F}^{-1}\left\{#1\right\}}
\newcommand{\FTc}[1]{{\cal F}{_C}\{#1\}}
\newcommand{\FTs}[1]{{\cal F}{_S}\{#1\}}
\newcommand{\periodic}[1]{\tilde{#1}}		% funcion periodica
\newcommand{\llave}[4]{ \left\{ \begin{array}{ll}
			#1 & #2 \\
			#3 & #4
			\end{array}
			\right.}
\newcommand{\da}{\longrightarrow}
\newcommand{\noda}{\longleftarrow}
\newcommand{\matlab}{Matlab}
\newcommand{\beq}{\begin{equation}}
\newcommand{\eeq}{\end{equation}}
\newcommand{\bseq}{\begin{subequations}}
\newcommand{\eseq}{\end{subequations}}

%\newcommand{\blin}{\begin{align}}
%\newcommand{\elin}{\end{align}}
%\newcommand{\bal}{\begin{align*}}
%\newcommand{\eal}{\end{align*}}
\newcommand{\baq}{\begin{eqnarray*}}
\newcommand{\eaq}{\end{eqnarray*}}
\newcommand{\eref}[1]{(\ref{#1})}
\newcommand{\ddelta}{{}^2\delta}
\newcommand{\trans}[1]{{#1}^{\ensuremath{\mathsf{T}}}}   % transpose

%       definicion de funciones step y shah
\newlength{\widthfontline}
\setlength{\widthfontline}{0.12ex}

\newcommand{\step}{\text{\mbox{\rule{0.3ex}{0cm}%
                        \rule{\widthfontline}{1.0ex}%
                        \hspace{-\widthfontline}%
                        \rule[1.0ex]{1.5ex}{\widthfontline}}}}
\newcommand{\shah}{\text{\mbox{\rule{0.3ex}{0cm}%
                        \rule{\widthfontline}{1.5ex}%
                        \rule{0.7ex}{\widthfontline}%
                        \rule{\widthfontline}{1.5ex}%
                        \rule{0.7ex}{\widthfontline}%
                        \rule{\widthfontline}{1.5ex}%
                        \hspace{\widthfontline}}}}
\newcommand{\sqsq}{\text{\mbox{\rule{0.3ex}{0cm}%
                        \rule{\widthfontline}{1.2ex}%
                        \rule[1.1ex]{0.7ex}{\widthfontline}%
                        \rule{\widthfontline}{1.2ex}%
                        \rule{0.7ex}{\widthfontline}%
                        \rule{\widthfontline}{1.2ex}%
                        \hspace{\widthfontline}}}}
%%%%%%%%%%%%%%%%%%%%%%%%%%%%%%%%%%%%%%%%%%%%%%%%%%%%%%%%%%%%%%
%%%%%%%%%%%%%%%%%%%%%%%%%%%%%%%%%%%%%%%%%%%%%%%%%%%%%%%%%%%%%%
%%%%%                  Funciones Propias                   %%
%%%%%%%%%%%%%%%%%%%%%%%%%%%%%%%%%%%%%%%%%%%%%%%%%%%%%%%%%%%%%%

\newcommand{\save}[2]{\newcommand{#1}{#2}} %funciona, pero hay que poner el backslash cuando se usa. Para llamarla hay que hacer \save{\#1}{#2}
\newcommand{\diff}[2]{\frac{d #1}{d #2}} % derivada
\newcommand{\vabs}[1]{\left\lvert #1 \right\rvert}%valor absoluto bonito
\newcommand{\corch}[1]{\left[ #1 \right]}
\newcommand{\squig}[1]{\left\lbrace #1 \right\rbrace}
\newcommand{\ptsis}[1]{\left( #1 \right)}
\newcommand{\overbar}[1]{\mkern 1.5mu\overline{\mkern-1.5mu#1\mkern-1.5mu}\mkern 1.5mu}
\newcommand{\logn}[1]{\text{log}\ptsis{#1}}%logaritmo natural
\newcommand{\ttt}[1]{\texttt{#1}}%formato computador, funciones de MATLAB
\newcommand{\qt}[1]{``{#1}''}%poner entre comillas
\newcommand{\tw}{\textwidth} %shortcut pal textiwdth
\newcommand{\mth}[1]{\( #1\)}
\newcommand{\mthd}[1]{\[ #1\]}
\newcommand{\defeq}{\overset{\small{\bigtriangleup}}{=}}
\newcommand{\bsh}{\textbackslash}
\newcommand{\intEval}[2]{\Big|_{#1}^{#2}} % Evaluate integral at limits
\newcommand{\atan}{\text{atan}}
\newcommand{\acos}{\text{acos}}
\newcommand{\asin}{\text{asin}}
\newcommand{\bs}[1]{\boldsymbol{#1}}
\newcommand{\mhr}[1]{$\text{m}^{#1}/\text{hr}$}
\usepackage{stackengine}
\def\delequal{\mathrel{\ensurestackMath{\stackon[2pt]{=}{\scriptscriptstyle\Delta}}}}
\newcommand{\fig}[5]
{
  \begin{figure}
    \centering
    \resizebox{#3}{#4}{\includegraphics{#1}}
    \caption{#5 \normalsize}
    \label{#2}
  \end{figure}
}

\newcommand{\rojo}[1]
{
{\color{red}#1}
}
%%%%%%%%%%%%%%%%%%%%%%%%%%%%%%%%%%%%%%%%%%%%%%%%%%%%%%%%%%%%%%
%% Manejo UTF8 de listings %%%%%%%%%%%%%%%%%%%%%%%%%%%%%%%%%
\graphicspath{{C:/Users/pablo/Ingenieria/Latex/}{./figuras/}}

\usepackage{float}
 
%\setlength\parindent{0pt} %esto hace la sangria nula

\begin{document}
\thispagestyle{empty}
%\vspace*{-1cm}
%\vspace*{-2cm}


\vspace*{-0.2cm}
\begin{center}
{\Large\bf El Juego en Chile}\\
\vspace*{2mm}
{\Large Como dárselo vuelta}\\
%\vspace*{3mm}
{16 de enero de 2018}\\
\vspace*{1mm}
{\bf Engine - King - Harmony}\\
\vspace*{1mm}
\end{center}
\hrule\vspace*{2pt}\hrule
%%%%%%%%%%%%%%%%%%%%%%%%%%%%%%
%%%%%%%%% ENCABEZADO %%%%%%%%%
%%%%%%%%%%%%%%%%%%%%%%%%%%%%%%
%\newpage
\setcounter{page}{1}
%\tableofcontents
%\newpage

\tableofcontents
\newpage
\section{Introducción y Motivación}

A partir de un asado el 15 de enero de 2018 y una conversación que llevó casualmente a hablar de temas de discotheques, mujeres y como seducir, fue claro que habían demasiados problemas en como se acerca un hombre chileno a las mujeres. La sociedad en la que vivimos (y por sociedad nos estamos refiriendo a Chile prácticamente) mucho del proceso de obtención de mujeres está asociado a la pinta o a la plata. Más aún, las discotheques - que son en general uno de los ambientes más prolíferos en mujeres - son distintas en todo los lugares del mundo y Chile no es la excepción. Chile cuenta con sus propias reglas y protocolos, los hombres tienden a cometer los mismos errores siempre y por eso muchas veces El Juego se transforma en El Azar.

El fin de este documento es dar vuelta esa situación. Principalmente, postulamos que es posible conseguir a \textbf{cualquier mujer} (independiente de su belleza física, inteligencia, \textbf{soltera o no}, etc) mediante el ejercicio y el análisis. Nuestras profesiones y aptitudes personales nos han llevado a darnos cuenta (por lo menos eso postula Pablo) que todo el sistema se puede vencer mediante \textbf{La Estrategia}:

\begin{itemize}
\item Estudio y adquisición de conocimiento
\item Iteración (pruebas en terreno)
\item Registro de resultados
\item Análisis de resultados
\item Modificación de estrategia
\item Práctica
\end{itemize}

Basamos La Estrategia en el hecho de que con método, disciplina y análisis es posible entender cualquier sistema y usarlo para el beneficio personal. En ningún caso esto busca denostar a la mujeres ni nada por el estilo. Solamente buscamos, dicho de forma muy simple, aumentar las probabilidades de éxito en la conquista y expandir el horizonte de mujeres a conseguir.

Basamos La Estrategia en \textbf{Los Axiomas}:
\begin{itemize}
\item Cualquier hombre puede conseguir a cualquier mujer.
\item Al igual que muchas habilidades del colegio o universidad, El Juego se puede enseñar, prácticar, pulir y mejorar.
\item A través de la honestidad personal, la crítica y la auto-crítica es posible continuamente mejorar el desempeño 
\item El rechazo de una mujer es completamente culpa del hombre y su forma de acercarse. En otras palabras, no hay excusas.
\item Queremos mujeres porque, últimamente, nos gustan mucho. Este documento no trata de denostarlas.

\item Todas y cada una de las reglas propuestas se pueden quebrar según corresponda. Sin embargo, estas tienden a funcionar mejor.

\end{itemize}

Muchas de las ideas de este escrito están contenidas o reflejadas en el libro \textit{The Game} de Niel Strauss. Es requisito leer el libro para poder entender la motivación detrás de estas ideas así como para constatar los resultados y el poder de una estrategia correcta. Muchos lectores del libro se quedan en los consejos prácticos y útiles de este (que lo son, sin duda alguna). El verdadero poder del libro está en entender el funcionamiento mental del protagonista y su dedicación. Es decir, lo correcto es leer entre líneas, obviamente aprovechando el potencial de las ideas prácticas ahí expuestas.

La distribución de este documento, así como la inclusión en los grupos de discusión en redes o el conocimiento de que esta operación existe está \textbf{absoultamente bajo el criterio de los participantes actuales}. Algunos de los riesgos de distribución de este material o estas ideas a terceros son:

\begin{itemize}
\item Bullying hasta la muerte.
\item Destrucción de cada uno de los intentos de jugar El Juego debido a la aparición del tercero.
\item Copiones.
\end{itemize}

Por lo tanto, en caso de querer introducir a alguien \textbf{El Equipo}, se debe consultar previamente entre los miembros actuales. En otras palabras, esto es un secreto total entre los miembros de El Equipo e introducir a un tercero - o incluso mencionarle que esto está sucediendo - debe ser aprobado por todo El Equipo. 

El documento está en continuo desarrollo y recopila información, ideas y resultados de distintas estrategias probadas. A medida que el documento avance se evaluará distintas formas de ordenarlo según vaya siendo lógico.

\section{Resumen de \textit{The Game} \label{sec:TheGame}}
\section{Procedimiento General \label{sec:procedimientoGeneral}}
{\color{red} Sería bueno poner acá una lista concreta o un diagrama de bloques de como los macro pasos a seguir del proceso}.

\subsection{Trabajo en Equipo \label{ssec:trabajoEnEquipo}}

Todas las técnicas y estrategias desarrolladas en este documento están pensadas para el uso individual o bien para el trabajo en Equipo. Sin embargo, la experiencia en terreno ha mostrado que el compañero puede ser un factor de gran ayuda para ganar El Juego.

Además, existen ciertas frases o jugadas que se pueden efectuar de mejor forma en equipo, o incluso solo en este caso. En particular, El Juego en El Baile debe ser hecho por lo menos de a 2 (ver sección \ref{sec:elBaile}). Debido a lo anterior, es altamente recomendable jugar en Equipo.

Sin embargo, como es de esperarse en cualquier equipo humano, existirán conflictos en caso de que no existan normas claras. Por ello, se proponen las siguientes reglas que deberán ser consensuadas antes de que se empieze a jugar:

\begin{itemize}
\item Los Objetivos para cada uno de los integrantes del Equipo deben estar claros \textbf{antes} de empezar el juego en un \textit{set}.
\item En caso de que uno de los miembros del Equipo desee cambiar de objetivo una vez empezado el Juego con un \textit{set}, puede hacerlo mientras su nuevo Objetivo no sea el Objetivo del otro miembro.
\item {\color{red}Regla 3 para asegurar éxito y convivencia}.
\item {\color{red}Regla 4 para asegurar éxito y convivencia}.
\item {\color{red}etc.}
\end{itemize}
\section{Primer contacto o aproximación}

Típicamente, una de las primeras etapas difíciles es la aproximación. La mejor manera de superar esta etapa - en situaciones conversacionales, pues El Baile tiene otros protocolos - es a través de frases, relatos, juegos y en definitiva \textbf{guiones pequeños} ya ensayados y probados.

Los problemas típicos en esta etapa son:
\begin{itemize}
\item Establecer contacto en un lapso de tiempo apropiado (\textit{three-second rule}).
\item Grupos de mujeres u hombres (revisar sección \ref{sec:TheGame} para ver porque esto es importante y a la vez difícil).
\item Conversación adecuada, es decir, no demasiado frontal pero tampoco aburrida ni típica.
\item Falta de confianza debido a \textbf{El Objetivo} (\textit{target} en inglés).
\end{itemize}

El último punto es obviamente uno de los más difíciles de resolver. De hecho, es uno de los problemas transversales a la hora de jugar El Juego. Sin embargo, a través de La Estrategia y la práctica, es posible con el tiempo obtener resultados que vayan poco a poco generando más confianza.

Uno de los principios fundamentales del acercamiento es que no debiese contener indicaciones de ningún tipo hacia algo físico o sexual inmediatamente. Eso viene en etapas posteriores del procedimiento. El objetivo del primer acercamiento es literalmente garantizar el poder empezar una conversación.

A continuación, se encuentra una lista detallada de algunas frases o guiones iniciales con resultados, problemas, ventajas y un análisis general de su efecto.

\subsection{\textit{Openers}\label{ssec:openers}}
Estos son pequeños comentarios o formas de acercarse y empezar una conversación. Todas estas ideas son literalmente la primera o primeras dos oraciones para decir y pueden incluir o no la respuesta de El Objetivo.

Obviamente no solo importan las palabras dichas. Es muy importante el lenguaje no verbal y paraverbal de estas presentaciones. Algunos comentarios respecto a esto son:
\begin{itemize}
\item Postura corporal respecto a El Objetivo angulada. Se debe hablar siempre como por el lado del hombro, como si en cualquier momento te vas a ir. Eso debería generar (suponiendo que se está generando el interés) que El Objetivo tenga que trabajar para que uno se quede.
\item Tono de voz adecuado. Que no sea confrontacional (\qt{A ver choriza, te apuesto que te lo adivino} no funciona, porque es confrontacional y le hace pensar en como se relacionan los hombres con sus amigos) pero tampoco payaso entretenedor. La mejor forma es honestamente cautivar el interés hacia la magia de lo que va a pasar (\qt{Te puedo mostrar algo interesante?}, \qt{El otro día leí algo interesante sobre esto. Te tinca si te lo muestro?}). {\color{red} Falta robustecer con pruebas en terreno}.
\item \textbf{Sonreír}.
\item \textbf{False-time constraint}. Entre el \textit{opener} poner una restricción de tiempo (\qt{me tengo que ir})
\end{itemize}

Además de lo anterior, muchas veces será necesario rellenar la conversación para mantener el interés, convencer de seguir conversando, etc. Lo importante es {\color{red} falta completar}.

Por último, se debe entender que existe un calce entre El Objetivo y el \textit{opener} más adecuado. Debido a que la aproximación debe ser muy rápida (3 segundos), es muy poco el tiempo para decidir cual es el correcto, por lo que esto requiere de instinto. El instinto lo desarrolla la experiencia y la práctica.

\subsubsection{Número del 1 al 10}
\textbf{Procedimiento}:
\begin{itemize}
\item Preguntar a una mujer que piense en un número del 1 al 10.
\item Adivinarlo. Con alta probabilidad (probada en terreno hasta el momento y según \textit{The Game}) el número es 7.
\end{itemize}
Las ventajas de esta entrada son principalmente su simpleza, su aparente inocuidad y su porcentaje de éxito.

Las desventajas aparecen en los problemas típicos.

\textbf{Problemas típicos}:
\begin{enumerate}

\item No adivinar el número.
\item \qt{A ver, hazme otro truco}.
\item \qt{Que fooome, te apuesto que funciona la mayoría de las veces}
\item \qt{Te apuesto que esta se la haci a todas las minas}\qt{Te la sabes por libro}
\end{enumerate}

\textbf{Soluciones posibles}

\begin{enumerate}
\item Por el momento, no hay solución comprobada y definitiva. Una buena salida es reconocer la derrota (\qt{No pude ver a través de tu mente, me has derrotado}, \qt{Solo funciona con las mentes simples {\color{red}No ha sido probada, pero tiene buena pinta}}) o bien cambiar el eje de la conversación (\qt{Cierto que no funcionó, pero ahora podemos conversar de cosas más interesantes.}{\color{red} Falta comprobar en terreno})
\item Esto quiere decir que por el momento no hay ningún interés físico por parte de El Objetivo, si no que eres el entretenimiento de la noche. Rápidamente, esto tiene que ser eliminado del imaginario mediante cosas del estilo \qt{Chaaa, ¿se supone que estoy para entretenerte?}, \qt{Hmm, no me parece. Haz tu algo interesante} ({\color{red}Esta es de choros, pero no ha sido probada, quizás no sea buena}). Una buena idea es seguir con un comentario negativo o \textit{neg} (revisar sección \ref{sec:TheGame}).
\item {\color{red}Falta generar algo certero, además no pasa demasiado}
\item Esta es difícil y es culpa probablemente de como se dijo el \textit{opener} o de que El Objetivo era equivocado para este \textit{opener} en particular. Quizás una buena idea es reconocer eso+\textit{neg}. O bien reconocer el plan final (\qt{Quizás, pero lo bueno es que ahora estamos conversando. Y no puedes culparme por querer conversar contigo}.) {\color{red} Difícil de salir jugando de esta, requiere diseño e iteraciones}.
\end{enumerate}
\subsubsection{El nombre de mi sobrina}

\textbf{Procedimiento}
\begin{itemize}
\item Acercarse a un grupo con el pretexto \qt{Ustedes se ven como un grupo inteligente}, \qt{Necesito una opinión femenina}.
\item \qt{Tengo una sobrina que acaba de nacer y le quieren poner Almendra. ¿Qué opinan?}
\end{itemize}

\textbf{Hasta ahora resulta infalible}.

\subsubsection{El nombre de los perritos }

\subsubsection{Apuesta con un amigo}

\subsubsection{El best-friend game}

\subsubsection{Hey, ven, te echo de menos}

\subsubsection{Ex-Girlfriend Question}

\subsubsection{Me puedes sacar una foto?}
Utilizable como neg (al Objetivo bueno la ninguneas).

\subsubsection{Quien miente más: los hombres o las mujeres?}
Ideal con una wingwoman.

Nivel 2: usar al Objetivo para abrir otros grupos.

\subsubsection{Vieron la pelea de mujeres afuera?}
\subsubsection{\textit{Opener} 3}
\subsection{Grupos de personas \label{ssec:gruposDePersonas}}
{\color{red} Existe experiencia, pero falta documentarla y analizarla a cabalidad.}
\subsection{Técnica 2 respecto a primer contacto}
\subsection{Técnica 3 respecto a primer contacto}

\section{Transición \label{sec:transicion}}
Posteriormente al primer acercamiento, se deberá rellenar la conversación para poder ir acercándose poco a poco a la etapa de transición. Algunos comentarios acerca del relleno son:

\begin{itemize}
\item Mantener la postura corporal angulada.
\item Evitar comentarios acerca de la belleza física.
\item Uso de \textit{negs} inteligentes y asociados a la belleza física. \qt{Croe que ese polerón te hace ver como una niña pequeña jajaja}, \qt{Es la segunda vez que veo ese vestido hoy día y me gusta bastante}. {\color{red} Este es un campo muy grande}
\item {\color{red}Comentario importante acerca del relleno}
\item {\color{red}Comentario importante acerca del relleno}
\end{itemize}
El relleno debería durar entre 6 y 10 minutos y posteriormente se debe generar una transición.
\subsection{Negs\label{sec:negs}}

\subsubsection{Ojos turnios}
\textbf{\qt{Sabes qué? Tienes uno ojos lindos pero eres un poco turnia.}}

\textbf{Ventajas}
\begin{itemize}
\item Simple de memorizar.
\item En general a las mujeres se les dice mucho que tienen los ojos lindos, por lo que esto es algo conocido y a la vez innovador.
\item No es demasiado ofensivo.
\end{itemize}

\textbf{Desventajas}
\begin{itemize}
\item Si El Objetivo es turnio puede ser ofensivo.
\end{itemize}

\subsubsection{Piercing extraño}
{\color{red}King rellenar con la frase completa, ventajas y desventajas}

\subsubsection{Mosca encima}
\begin{itemize}
\item \qt{Se te acaba de parar una mosca encima. No te muevas!}
\item La golpeas un poco.
\item \qt{Te duchaste? Mira como se te paran las moscas}
\end{itemize}

\subsubsection{Me acabas de escupir}

\subsubsection{Te lavaste los dientes?}
\subsection{Movimientos de Transición \label{ssec:movimientosTransicion}}

\subsubsection{Las tres preguntas}

\textbf{Procedimiento}
\begin{itemize}
\item Decirle a El Objetivo que se vienen tres pregunas. {\color{red} La experiencia dicta que sí hay que hacer esto, pero faltan iteraciones}
\item Estás soltera? Si sí, pasar a la próxima. Si no {\color{red}falta documentar y probar}
\item Me encuentras mino/guapo/apuesto/etc? Si sí, pasar a la próxima. Si no {\color{red}falta diseñar respuesta, documentar y probar}
\item Esta se debe decir de forma casi textual: \qt{Entonces cuál es tu excusa para no darme un beso?}.
\end{itemize}
Entre pregunta y pregunta probablemente haya que rellenar un poco: algunos comentarios del estilo \qt{A yaa que buena} o cosas así. A veces funciona incluso mantener la tensión y no decir nada más hasta que El Objetivo pida la próxima pregunta (por ejemplo, en ambientes de música electrónica se puede mantener muy alto el nivel de tensión entre pregunta y pregunta con mantener el silencio).

Esta técnica tiene sus ventajas y desventajas extremadamente claras. Los problemaas aparecen en las negativas a las respuestas y serán resueltos a futuro.

\subsubsection{La pregunta única}
Esta técnica está directamente sacada del libro \textit{The Game}.

\textbf{Procedimiento}
\begin{itemize}
\item \qt{Me quieres dar un beso?}
\item Si sí, si. Escaso.
\item Si \qt{no sé}, \qt{A ver, cachemos } y tú lo das. Más común de lo que parece.
\item Si no, \qt{No dije que podías, solo te pregunté porque te vi pensando en algo}. 
\end{itemize}

\subsubsection{Técnica número 3 para la transición}

\subsubsection{Técnica número 4 para la transición}

\subsection{Estrategia 2 para la transición}
\subsection{Estrategia 3 para la transición}
\section{El Baile \label{sec:elBaile}}
En Chile en particular, la mayoría de los ambientes con mujeres involucran bailar. Esta es quizás una de las motivaciones más importantes para hacer un documento distinto a \textit{The Game}, donde la mayoría de las interraciones ocurren en bares o por lo menos en el bar de una discotheque.

Parte de El Juego en Chile consiste en bailar. Es necesario saber bailar, por lo que algún miembro de El Equipo que se sienta en desventaja en esta área no debería tener verguenza en tomar clases de baile. El baile es una habilidad relativamente transversal - si sabes bailar un género bien, puedes bailar relativamente cualquiera - pero se recomiendan ojalá clases de baile de música tropical. En Chile se baila predominantemente reggaetón o reggaetón cumbia, que es una música hipersexualizada. {\color{red} Habría que revisar si esto es bueno o malo, es decir, si El Equipo está bailando de forma correcta o está cayendo en la trampa de la música}

En cualquier caso, bailar tiende a ser entretenido. El problema es, obviamente, el acercamiento a bailar y el porcentaje de éxito que se tiene en esta etapa al bailar. Se propone en este documento una estrategia especial para el baile.

\subsection{Razonamiento y Axiomas\label{ssec:razonamiento}}

En un equipo de fútbol existen arqueros, defensas, etc. Nadie espera que el arquero meta goles ni que el delantero ataje penales. Lo mismo debiese pasar en un equipo de baile. En su forma más sencilla, el equipo de baile consta de dos personas. Lo que se debe hacer es desarrollar una estrategia en que cada integrante del equipo cumpla un rol específico.

Principalmente, se debe evitar \textbf{el tiempo muerto mirando por la discotheque.} Es decir, la decisión de sacar a bailar una mujer debe ser tomada en tiempo real y no haciendo \textit{scouting}. Además, se debe evitar a toda costa la \textbf{deliberación}, que se entiendo como el proceso en que dos hombres se hacen gestos o conversan acerca de si se \qt{apañan} a sacar a bailar a dos objetivos. Usualmente esto sucede a menos de 3 metros de los objetivos en cuestión, por lo que es detectado por ellas y rápidamente baja la probabilidad de éxito.

En términos ideales, El Equipo debería ser capaz de fluir por la pista de baile y automáticamente detectar, acercarse, sacar a bailar y posteriormente transicionar con cualquier grupo de mujeres. Para lo anterior, se tienen los siguientes principios o axiomas:

\begin{itemize}
\item El Baile parte desde el momento en que se entra a la pista de baile. Es decir, mientras se camina por ella se debe estar bailando, sonríendo, en fin, teniendo actitud de baile.
\item La mejor forma de ordenar tácticamente el equipo es en fila. Esto permite una distribución de roles adecuada y funcional. {\color{red} Falta corroborar que para Equipos de más de dos personas esto sea cierto}
\item Las posibilidades de éxito son óptimas en un equipo de dos personas. {\color{red} Habría que revisar esto en detalle, pero pareciera que sacar a bailar solo es muchísimo más complejo}.
\item El que va a adelalte será \textbf{Los Ojos}. Es decir, es su deber decidir que pareja o grupo de objetivos es el correcto.
\item El resto de El Equipo \textbf{no mira la pista de baile}. Mira - con actitud correcta y siempre moviéndose - la espalda de Los Ojos o del miembro que tenga al frente. Debido a que Los Ojos están mirando y están abiertamente moviéndose, los miembros de atrás están protegidos de la pista, por o que pueden mirar muy concentradamente la espalda de la persona que está adelante.
\item El principio fundamental de El Equipo es que se hace lo mejor para El Equipo. Es decir, cuando Los Ojos eligen, eligen para todo el equipo y no solo para él. Esto elimina que alguien de El Equipo quede bailando con un Objetivo poco deseable. {\color{red} Habría que hacer alguna cláusula o código para pedir un \qt{apáñame con la gorda}}.
\item Mediante un diccionario y códigos gestuales - hechos en la espalda de los integrantes de El Equipo - es posible eliminar toda el \textit{scouting} y deliberación típicas de las situaciones con pista de baile.
\item El acercamiento a las mujeres \textbf{siempre} debe ser de forma lateral angulada. Es decir, \textbf{nunca} se debe acercar a un Objetivo por detrás (evitando así la clásica dada vuelta) o por delante (demasiado confrontacional). El acercamiento es lateral como formando la cara 4 de un dado.
\item Según todo esto - y si es bien llevado a cabo de forma correcta - será posible eliminar el clásico \qt{Quieres bailar?}. Esa frase, o ese tipo de frases, están planteadas desde su origen de forma derrotista porque admiten rápidamente una respuesta negativa. 
\end{itemize} 

\subsection{Código de Baile\label{ssec:codigoBaile}}
\begin{itemize}
\item \textbf{Subirse la camisa o polera en la espalda con la mano derecha/izquierda}: En la próxima pareja de mujeres que aparezca lateralmente a la derecha/izquierda, Los Ojos bailará con la mujer más lejana.
\item \textbf{Tropiezo leve hacia la derecha/izquierda}: En la próxima pareja de mujeres que aparezca lateralmente a la derecha/izquierda, Los Ojos bailará con la mujer más cerca.
\item \textbf{Cancelación de instrucción anterior}: {\color{red}Falta definir alguno}
\end{itemize}

\subsection{Invitación verbal a bailar \label{ssec:invitacionABailar}}
{\color{red}Falta inventar frases, probarlas, iterarlas, etc. En general deberían estar todas orientadas a la misma filosofía del \textit{neg}}.
\subsection{De Baile a algo más \label{ssec:baileAMas}}
{\color{red} Esta sección se refiere a como concretar. Acá entran consideraciones propias de El Objetivo: si es muy lanzada, si está medio borracha, etc. Pero sin lugar a dudas a lo que hay que apuntar acá es a SACAR AL OBJETIVO DE LA PISTA DE BAILE }.

\section{El Juego en Whatsapp \label{sec:whatsapp}}
{\color{red}Falta completar}
\section{Recomendaciones y Consejos Prácticos\label{sec:tipsFinales}}
\begin{itemize}
\item Evitar dar el nombre. Lo que hay que lograr es que El Objetivo lo pregunte directamente (lo que constituye un IOI según lo comentado en la sección \ref{sec:TheGame}).
\item Cuando se está solo en una discotheque, lo que más se tiene que evitar es estar con cara de solo o buscando un Objetivo. La recomendación es \textbf{hacer como si se hablara por teléfono mientras se camina, como si estuvieras buscando a la persona con la que hablas}. Lo anterior es una excelente técnica: permite ver libremente e identificar objetivos, te hace ver socialmente importante y en caso de encontrar un objetivo (cruzar miradas) permite hacer el show de decirle al interlocutor falso del teléfono \qt{Compadre, te tengo que dejar en este minuto}. Si lo anterior se hace bien, El Objetivo lo verá. {\color{red} Esta no ha sido probada pero se ve demasiado buena.}
\item Tip 3
\item Tip 4
\end{itemize}
\end{document}