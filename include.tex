% define tabular espaciado

\newenvironment{mitab}[2]{\vspace{#1}\begin{tabular}{#2}}%
		{\end{tabular}\vspace{5mm}}
% define mi tipo de frames
\newenvironment{myFrame}[1]{
\begin{frame}[allowframebreaks]
\frametitle{#1}
\justifying
}
{\end{frame}}

%\newenvironment{myEqn}{\begin{subequations}\begin{gather}}
%{\end{gather}\end{subequations}}
% define listas

\newcommand{\be}{\begin{enumerate}}
\newcommand{\ee}{\end{enumerate}}
\newcommand{\bi}{\begin{itemize}}
\newcommand{\ei}{\end{itemize}}
\newcommand{\bc}{\begin{center}}
\newcommand{\ec}{\end{center}}

% puntaje pruebas
\newcommand{\pts[1]}{{\color{red} $\to$ \boxed{\textbf{#1 pts.}}}}

% cosas matematicas

\renewcommand{\d}{\mathsf{d}}
\newenvironment{eqn}{\begin{eqnarray*}}{\end{eqnarray*}}

\newcommand{\re}[1]{\mbox{Re\{$#1$\}}}		% la parte real
\newcommand{\im}[1]{\mbox{Im\{$#1$\}}}		% la parte imaginaria
\newcommand{\gauss}{\operatorname{Gauss}}		% gauss
\newcommand{\rect}{\sqcap}			% rect
\newcommand{\pillbox}{\operatorname{circ}}		% circ
\newcommand{\sinc}{\operatorname{sinc}}		% sinc
\newcommand{\jinc}{\operatorname{jinc}}		% jinc
\newcommand{\asinc}{\operatorname{asinc}}		% asinc
\newcommand{\sen}{\operatorname{sen}}		% seno
\newcommand{\senh}{\operatorname{senh}}		% seno
\newcommand{\triang}{\wedge}			% triangulo
\newcommand{\allint}{\ds{\int_{-\infty}^\infty}}	% integral de -infinito a +infinito
\newcommand{\midint}{\int_{0}^\infty}		% integral de cero a +infinito
\newcommand{\allsum}[1]{\sum_{#1=-\infty}^\infty} % idem suma
\newcommand{\hor}{\operatorname{\uparrow\uparrow}}		% horquilla
\newcommand{\ahor}{\operatorname{\uparrow\downarrow}}		% anti horquilla
\newcommand{\sgn}{\operatorname{sgn}}			% signo
\newcommand{\ddx}{\frac{d}{dx}}			% d/dx
\newcommand{\ddt}{\frac{d}{dt}}			% d/dt
\newcommand{\partiald}[2]{\frac{\partial#1}{\partial#2}}		% d/dx (parcial)
\newcommand{\der}[1]{#1^{\prime}}		% ' derivada
\newcommand{\derr}[1]{#1^{\prime\prime}}	% '' doble derivada
\newcommand{\FT}[1]{{\cal F}\left\{#1\right\}}		% FT
\newcommand{\IFT}[1]{{\cal F}^{-1}\left\{#1\right\}}
\newcommand{\FTc}[1]{{\cal F}{_C}\{#1\}}
\newcommand{\FTs}[1]{{\cal F}{_S}\{#1\}}
\newcommand{\periodic}[1]{\tilde{#1}}		% funcion periodica
\newcommand{\llave}[4]{ \left\{ \begin{array}{ll}
			#1 & #2 \\
			#3 & #4
			\end{array}
			\right.}
\newcommand{\da}{\longrightarrow}
\newcommand{\noda}{\longleftarrow}
\newcommand{\matlab}{Matlab}
\newcommand{\beq}{\begin{equation}}
\newcommand{\eeq}{\end{equation}}
\newcommand{\bseq}{\begin{subequations}}
\newcommand{\eseq}{\end{subequations}}

%\newcommand{\blin}{\begin{align}}
%\newcommand{\elin}{\end{align}}
%\newcommand{\bal}{\begin{align*}}
%\newcommand{\eal}{\end{align*}}
\newcommand{\baq}{\begin{eqnarray*}}
\newcommand{\eaq}{\end{eqnarray*}}
\newcommand{\eref}[1]{(\ref{#1})}
\newcommand{\ddelta}{{}^2\delta}
\newcommand{\trans}[1]{{#1}^{\ensuremath{\mathsf{T}}}}   % transpose

%       definicion de funciones step y shah
\newlength{\widthfontline}
\setlength{\widthfontline}{0.12ex}

\newcommand{\step}{\text{\mbox{\rule{0.3ex}{0cm}%
                        \rule{\widthfontline}{1.0ex}%
                        \hspace{-\widthfontline}%
                        \rule[1.0ex]{1.5ex}{\widthfontline}}}}
\newcommand{\shah}{\text{\mbox{\rule{0.3ex}{0cm}%
                        \rule{\widthfontline}{1.5ex}%
                        \rule{0.7ex}{\widthfontline}%
                        \rule{\widthfontline}{1.5ex}%
                        \rule{0.7ex}{\widthfontline}%
                        \rule{\widthfontline}{1.5ex}%
                        \hspace{\widthfontline}}}}
\newcommand{\sqsq}{\text{\mbox{\rule{0.3ex}{0cm}%
                        \rule{\widthfontline}{1.2ex}%
                        \rule[1.1ex]{0.7ex}{\widthfontline}%
                        \rule{\widthfontline}{1.2ex}%
                        \rule{0.7ex}{\widthfontline}%
                        \rule{\widthfontline}{1.2ex}%
                        \hspace{\widthfontline}}}}
%%%%%%%%%%%%%%%%%%%%%%%%%%%%%%%%%%%%%%%%%%%%%%%%%%%%%%%%%%%%%%
%%%%%%%%%%%%%%%%%%%%%%%%%%%%%%%%%%%%%%%%%%%%%%%%%%%%%%%%%%%%%%
%%%%%                  Funciones Propias                   %%
%%%%%%%%%%%%%%%%%%%%%%%%%%%%%%%%%%%%%%%%%%%%%%%%%%%%%%%%%%%%%%

\newcommand{\save}[2]{\newcommand{#1}{#2}} %funciona, pero hay que poner el backslash cuando se usa. Para llamarla hay que hacer \save{\#1}{#2}
\newcommand{\diff}[2]{\frac{d #1}{d #2}} % derivada
\newcommand{\vabs}[1]{\left\lvert #1 \right\rvert}%valor absoluto bonito
\newcommand{\corch}[1]{\left[ #1 \right]}
\newcommand{\squig}[1]{\left\lbrace #1 \right\rbrace}
\newcommand{\ptsis}[1]{\left( #1 \right)}
\newcommand{\overbar}[1]{\mkern 1.5mu\overline{\mkern-1.5mu#1\mkern-1.5mu}\mkern 1.5mu}
\newcommand{\logn}[1]{\text{log}\ptsis{#1}}%logaritmo natural
\newcommand{\ttt}[1]{\texttt{#1}}%formato computador, funciones de MATLAB
\newcommand{\qt}[1]{``{#1}''}%poner entre comillas
\newcommand{\tw}{\textwidth} %shortcut pal textiwdth
\newcommand{\mth}[1]{\( #1\)}
\newcommand{\mthd}[1]{\[ #1\]}
\newcommand{\defeq}{\overset{\small{\bigtriangleup}}{=}}
\newcommand{\bsh}{\textbackslash}
\newcommand{\intEval}[2]{\Big|_{#1}^{#2}} % Evaluate integral at limits
\newcommand{\atan}{\text{atan}}
\newcommand{\acos}{\text{acos}}
\newcommand{\asin}{\text{asin}}
\newcommand{\bs}[1]{\boldsymbol{#1}}
\newcommand{\mhr}[1]{$\text{m}^{#1}/\text{hr}$}
\usepackage{stackengine}
\def\delequal{\mathrel{\ensurestackMath{\stackon[2pt]{=}{\scriptscriptstyle\Delta}}}}
\newcommand{\fig}[5]
{
  \begin{figure}
    \centering
    \resizebox{#3}{#4}{\includegraphics{#1}}
    \caption{#5 \normalsize}
    \label{#2}
  \end{figure}
}

\newcommand{\rojo}[1]
{
{\color{red}#1}
}
%%%%%%%%%%%%%%%%%%%%%%%%%%%%%%%%%%%%%%%%%%%%%%%%%%%%%%%%%%%%%%
%% Manejo UTF8 de listings %%%%%%%%%%%%%%%%%%%%%%%%%%%%%%%%%