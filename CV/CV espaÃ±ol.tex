\documentclass[a4paper,10pt]{article}

\usepackage{amsmath}    % need for subequations
\usepackage{graphicx}   % need for figures
\usepackage{verbatim}   % useful for program listings
\usepackage{color}      % use if color is used in text
\usepackage{subfigure}  % use for side-by-side figures
\usepackage{hyperref}   % use for hypertext links, including those to external documents and URLs
\usepackage[spanish]{babel}
\usepackage[utf8]{inputenc}
\usepackage{fullpage}
\usepackage[section]{placeins} %esto evita que imagenes se salgan de su seccion
\usepackage{cite} %referencias
\usepackage{hyperref}



%Setup hyperref package, and colours for links
\usepackage{hyperref}
\definecolor{linkcolour}{rgb}{0,0.2,0.6}
\hypersetup{colorlinks,breaklinks,urlcolor=linkcolour, linkcolor=linkcolour}

\usepackage{supertabular} 				%for Grades
\usepackage{titlesec}					%custom \section
%CV Sections inspired by: 
%http://stefano.italians.nl/archives/26
\titleformat{\section}{\Large\scshape\raggedright}{}{0em}{}[\titlerule]
\titlespacing{\section}{0pt}{3pt}{3pt}

%FONTS
%\defaultfontfeatures{Mapping=tex-text}
%\setmainfont[SmallCapsFont = Fontin SmallCaps]{Fontin}
%%% modified for Karol Kozioł for ShareLaTeX use


\usepackage{float}
\usepackage[utf8]{inputenc}
\usepackage[T1]{fontenc}

\newcommand{\braros}[1]{\left\lbrace #1 \right\rbrace}
\newcommand{\bred}[1]{\left( #1 \right)}
\newcommand{\bcuad}[1]{\left[ #1 \right]}
\newcommand{\e}[1]{\cdot 10^{#1}}
\newcommand{\derp}[2]{\dfrac{\partial #1}{\partial #2}}
\renewcommand{\sin}{\operatorname{sen}}
\renewcommand{\arcsin}{\operatorname{arcsen}}
\newcommand{\qt}[1]{``{#1}''}%poner entre comillas

\setlength\parindent{0pt} %esto hace la sangria nula
%% define tabular espaciado

\newenvironment{mitab}[2]{\vspace{#1}\begin{tabular}{#2}}%
		{\end{tabular}\vspace{5mm}}
% define mi tipo de frames
\newenvironment{myFrame}[1]{
\begin{frame}[allowframebreaks]
\frametitle{#1}
\justifying
}
{\end{frame}}

%\newenvironment{myEqn}{\begin{subequations}\begin{gather}}
%{\end{gather}\end{subequations}}
% define listas

\newcommand{\be}{\begin{enumerate}}
\newcommand{\ee}{\end{enumerate}}
\newcommand{\bi}{\begin{itemize}}
\newcommand{\ei}{\end{itemize}}
\newcommand{\bc}{\begin{center}}
\newcommand{\ec}{\end{center}}

% puntaje pruebas
\newcommand{\pts[1]}{{\color{red} $\to$ \boxed{\textbf{#1 pts.}}}}

% cosas matematicas

\renewcommand{\d}{\mathsf{d}}
\newenvironment{eqn}{\begin{eqnarray*}}{\end{eqnarray*}}

\newcommand{\re}[1]{\mbox{Re\{$#1$\}}}		% la parte real
\newcommand{\im}[1]{\mbox{Im\{$#1$\}}}		% la parte imaginaria
\newcommand{\gauss}{\operatorname{Gauss}}		% gauss
\newcommand{\rect}{\sqcap}			% rect
\newcommand{\pillbox}{\operatorname{circ}}		% circ
\newcommand{\sinc}{\operatorname{sinc}}		% sinc
\newcommand{\jinc}{\operatorname{jinc}}		% jinc
\newcommand{\asinc}{\operatorname{asinc}}		% asinc
\newcommand{\sen}{\operatorname{sen}}		% seno
\newcommand{\senh}{\operatorname{senh}}		% seno
\newcommand{\triang}{\wedge}			% triangulo
\newcommand{\allint}{\ds{\int_{-\infty}^\infty}}	% integral de -infinito a +infinito
\newcommand{\midint}{\int_{0}^\infty}		% integral de cero a +infinito
\newcommand{\allsum}[1]{\sum_{#1=-\infty}^\infty} % idem suma
\newcommand{\hor}{\operatorname{\uparrow\uparrow}}		% horquilla
\newcommand{\ahor}{\operatorname{\uparrow\downarrow}}		% anti horquilla
\newcommand{\sgn}{\operatorname{sgn}}			% signo
\newcommand{\ddx}{\frac{d}{dx}}			% d/dx
\newcommand{\ddt}{\frac{d}{dt}}			% d/dt
\newcommand{\partiald}[2]{\frac{\partial#1}{\partial#2}}		% d/dx (parcial)
\newcommand{\der}[1]{#1^{\prime}}		% ' derivada
\newcommand{\derr}[1]{#1^{\prime\prime}}	% '' doble derivada
\newcommand{\FT}[1]{{\cal F}\left\{#1\right\}}		% FT
\newcommand{\IFT}[1]{{\cal F}^{-1}\left\{#1\right\}}
\newcommand{\FTc}[1]{{\cal F}{_C}\{#1\}}
\newcommand{\FTs}[1]{{\cal F}{_S}\{#1\}}
\newcommand{\periodic}[1]{\tilde{#1}}		% funcion periodica
\newcommand{\llave}[4]{ \left\{ \begin{array}{ll}
			#1 & #2 \\
			#3 & #4
			\end{array}
			\right.}
\newcommand{\da}{\longrightarrow}
\newcommand{\noda}{\longleftarrow}
\newcommand{\matlab}{Matlab}
\newcommand{\beq}{\begin{equation}}
\newcommand{\eeq}{\end{equation}}
\newcommand{\bseq}{\begin{subequations}}
\newcommand{\eseq}{\end{subequations}}

%\newcommand{\blin}{\begin{align}}
%\newcommand{\elin}{\end{align}}
%\newcommand{\bal}{\begin{align*}}
%\newcommand{\eal}{\end{align*}}
\newcommand{\baq}{\begin{eqnarray*}}
\newcommand{\eaq}{\end{eqnarray*}}
\newcommand{\eref}[1]{(\ref{#1})}
\newcommand{\ddelta}{{}^2\delta}
\newcommand{\trans}[1]{{#1}^{\ensuremath{\mathsf{T}}}}   % transpose

%       definicion de funciones step y shah
\newlength{\widthfontline}
\setlength{\widthfontline}{0.12ex}

\newcommand{\step}{\text{\mbox{\rule{0.3ex}{0cm}%
                        \rule{\widthfontline}{1.0ex}%
                        \hspace{-\widthfontline}%
                        \rule[1.0ex]{1.5ex}{\widthfontline}}}}
\newcommand{\shah}{\text{\mbox{\rule{0.3ex}{0cm}%
                        \rule{\widthfontline}{1.5ex}%
                        \rule{0.7ex}{\widthfontline}%
                        \rule{\widthfontline}{1.5ex}%
                        \rule{0.7ex}{\widthfontline}%
                        \rule{\widthfontline}{1.5ex}%
                        \hspace{\widthfontline}}}}
\newcommand{\sqsq}{\text{\mbox{\rule{0.3ex}{0cm}%
                        \rule{\widthfontline}{1.2ex}%
                        \rule[1.1ex]{0.7ex}{\widthfontline}%
                        \rule{\widthfontline}{1.2ex}%
                        \rule{0.7ex}{\widthfontline}%
                        \rule{\widthfontline}{1.2ex}%
                        \hspace{\widthfontline}}}}
%%%%%%%%%%%%%%%%%%%%%%%%%%%%%%%%%%%%%%%%%%%%%%%%%%%%%%%%%%%%%%
%%%%%%%%%%%%%%%%%%%%%%%%%%%%%%%%%%%%%%%%%%%%%%%%%%%%%%%%%%%%%%
%%%%%                  Funciones Propias                   %%
%%%%%%%%%%%%%%%%%%%%%%%%%%%%%%%%%%%%%%%%%%%%%%%%%%%%%%%%%%%%%%

\newcommand{\save}[2]{\newcommand{#1}{#2}} %funciona, pero hay que poner el backslash cuando se usa. Para llamarla hay que hacer \save{\#1}{#2}
\newcommand{\diff}[2]{\frac{d #1}{d #2}} % derivada
\newcommand{\vabs}[1]{\left\lvert #1 \right\rvert}%valor absoluto bonito
\newcommand{\corch}[1]{\left[ #1 \right]}
\newcommand{\squig}[1]{\left\lbrace #1 \right\rbrace}
\newcommand{\ptsis}[1]{\left( #1 \right)}
\newcommand{\overbar}[1]{\mkern 1.5mu\overline{\mkern-1.5mu#1\mkern-1.5mu}\mkern 1.5mu}
\newcommand{\logn}[1]{\text{log}\ptsis{#1}}%logaritmo natural
\newcommand{\ttt}[1]{\texttt{#1}}%formato computador, funciones de MATLAB
\newcommand{\qt}[1]{``{#1}''}%poner entre comillas
\newcommand{\tw}{\textwidth} %shortcut pal textiwdth
\newcommand{\mth}[1]{\( #1\)}
\newcommand{\mthd}[1]{\[ #1\]}
\newcommand{\defeq}{\overset{\small{\bigtriangleup}}{=}}
\newcommand{\bsh}{\textbackslash}
\newcommand{\intEval}[2]{\Big|_{#1}^{#2}} % Evaluate integral at limits
\newcommand{\atan}{\text{atan}}
\newcommand{\acos}{\text{acos}}
\newcommand{\asin}{\text{asin}}
\newcommand{\bs}[1]{\boldsymbol{#1}}
\newcommand{\mhr}[1]{$\text{m}^{#1}/\text{hr}$}
\usepackage{stackengine}
\def\delequal{\mathrel{\ensurestackMath{\stackon[2pt]{=}{\scriptscriptstyle\Delta}}}}
\newcommand{\fig}[5]
{
  \begin{figure}
    \centering
    \resizebox{#3}{#4}{\includegraphics{#1}}
    \caption{#5 \normalsize}
    \label{#2}
  \end{figure}
}

\newcommand{\rojo}[1]
{
{\color{red}#1}
}
%%%%%%%%%%%%%%%%%%%%%%%%%%%%%%%%%%%%%%%%%%%%%%%%%%%%%%%%%%%%%%
%% Manejo UTF8 de listings %%%%%%%%%%%%%%%%%%%%%%%%%%%%%%%%%
%\graphicspath{{C:/Users/dcruz/Documents/Latex/}}
\usepackage{amssymb} %para signos raros

\newcommand{\vv}[1]{\text{\v{#1}}}
\begin{document}

\newcommand{\PUC}{\textsc{Pontificia Universidad Católica de Chile}}
%%%%%%%%%%%%%%%%%%%%%%%%%%%%%%
%%%%%%%%% ENCABEZADO %%%%%%%%%
%%%%%%%%%%%%%%%%%%%%%%%%%%%%%%

%--------------------TITLE-------------
\pagestyle{empty} % non-numbered pages
\par{\centering
		{\Huge{Pablo \textsc{Diaz Titelman}}
	}\par
	}
\begin{center}
Ingeniero Civil Electricista

Magíster en Ciencias de la Ingeniería

\PUC

+56 9 83600796

\href{mailto:pdiaz2@uc.cl}{\underline{pdiaz2@uc.cl}}
\end{center}

%\bigskip
\section*{Antecedentes Académicos}
\begin{tabular}{p{2 cm}|p{11cm}|p{2cm}}	
\textsc{Agosto} 2016 
- 
\textsc{Noviembre} 2018
&
\textbf{\PUC}

\textit{Magíster (c) en Ciencias de la Ingeniería, área de Ingeniería Eléctrica}

\begin{itemize}
	\item Tesis: ``Model Predictive Control Based on Machine Learning Techniques for Paste Tailing Production''
	\item Profesor supervisor: Aldo \textsc{Cipriano}.
	\item Defensa programada para mediados de Noviembre.
	\item 150 créditos aprobados y 0 reprobados.
	\item Promedio acumulado: 6,46.
	\item Ayudante Postgrado: Control Inteligente, Control Predictivo.
\end{itemize}

&
Santiago, 

Chile\\

&\\

\textsc{Marzo} 2012 
- \textsc{Octubre} 2018 &
\textbf{\PUC}

\textit{Estudiante de Ingeniería Civil Electricista, Major en Ingeniería Eléctrica, Minor de Profundidad en Electrónica y Telecomunicaciones}

\begin{itemize}
	\item Ingreso con Matrícula de Honor.
	\item 600 créditos aprobados y 0 reprobados.
	\item Promedio acumulado: 5,84.
	\item Lugar 32 de 694 en ranking académico de la promoción.
	\item Entre el 15\% mejor en examen de licenciatura 2016-1.
	\item Ayudante Pregrado: Cálculo I, Señales y Sistemas, Telecomunicaciones.
\end{itemize}

&
Santiago, 

Chile\\

&\\

1998 - 2011 &
\textbf{\textsc{Colegio Craighouse}}

\textit{Educación Básica y Media}
\begin{itemize}
	\item Realización de programa IB con 37/45 puntos.
	\item Nota de enseñanza media: 6,94.
	\item Mejor promedio académico de la promoción.
	\item Premio de Física y Filosofía en promoción.
\end{itemize}

&
Santiago, Chile
\end{tabular}

\bigskip
\section*{\textsc{Antecedentes Laborales}}

\begin{tabular}{p{2 cm}|p{11cm}|p{2cm}}

\textsc{Marzo} 2017 - \textsc{Marzo} 2017 &
\textbf{\textsc{Honeywell Chile S.A}}

\textit{Trabajo profesional jornada completa}

\begin{itemize}
	\item La actividad desarrollda en la práctica profesional fue sumamente exitosa. Dado lo anterior y posterior a una reunión con el cliente Codelco, se fijó un contrato de dos semanas con Honeywell para implementar 20 lazos más de CPM y así acoplarlos al 80\% de los lazos de control predictivo implementados.
\end{itemize}
&
Santiago, 

Chile
\\

&\\

\textsc{Diciembre} 2016 - \textsc{Febrero} 2017 &
\textbf{\textsc{Honeywell Chile S.A}}

\textit{Práctica profesional}

\begin{itemize}
	\item CPM es una herramienta cuyo fin es automatizar, estandarizar y monitorear los indicadores de desempeño de sus controladores. Este software se encontraba en desuso. El trabajo consistió en explorar, configurar, automatizar y describir en plenitud la herramienta y aplicarla a dos lazos de control predictivo en Mina División Ministro Hales y El Teniente. Parte del trabajo consistió en ejecutar la herramienta y automatizar su operación con un fuerte énfasis en la transferencia y documentación correcta. La práctica fue aprobada con nota Distinguida.
\end{itemize}
&
Santiago, 

Chile
\\

&\\



	

\textsc{Diciembre} 2015 &
\textbf{\textsc{Constructora Terrazo Ltda.}}

\textit{Práctica profesional obrera}

\begin{itemize}
	\item El trabajo consistió en asistir al técnico eléctrico y mano de obra no calificada en la construcción de un condominio de cuatro casas. La práctica fue aprobada con nota Distinguida.
\end{itemize}
&
Santiago, 

Chile
\\

&\\

\textsc{Junio} 2015 -
\textsc{Febrero} 2016  &

\textbf{\textsc{\textit{Staff} Plan Educativo Judaico}}

\textit{Directorio}

\begin{itemize}
	\item Encargado de la organización y logística de varios grupos de estudios y actividades, así como de la producción y organización de jornadas del universo completo de
participantes (500).
	\item Organización y ejecución de un viaje de estudio a Israel durante febrero. 
\end{itemize}
&
Santiago, 

Chile
\\

&\\

2010 -
2014 &

\textbf{\textsc{Movimiento juvenil \textit{Bet-El}}}

\textit{Dirigente y Directorio}

\begin{itemize}
	\item Encargado de la transmisión de conocimiento y valores judíos a través de educación no formal. Organización, financiamiento y gestión de variados proyectos: viajes, capacitaciones .
	\item Integrante del Directorio del movimiento los años 2012 y 2014.
	\item Actividad voluntaria y no remunerada.
\end{itemize}
&
Santiago, 

Chile
\\

&\\

\textsc{Marzo} 2013 - \textsc{Diciembre} 2013 &

\textbf{\textsc{Preuniversitario Mauro Quintana}}

\textit{Asistente de profesor}

\begin{itemize}
	\item El trabajo consistió en dictar las clases recuperativas y en responder dudas de los alumnos de forma remota.
\end{itemize}
&
Santiago, 

Chile
\end{tabular}

\bigskip
\section*{\textsc{Competencias Profesionales}}

\begin{tabular}{ll}
Control automático y sistemas inteligentes		& avanzado\\
Control predictivo & avanzado\\
Aprendizaje de máquinas		& avanzado\\
Control de procesos & medio\\
Electrónica analógica y digital & medio\\
Telecomunicaciones		& medio\\
Máquinas eléctricas		& medio\\
Big Data Analytics & medio\\
Tecnologías de información & medio\\
Mercados eléctricos		& básico\\
Sistemas de potencia & básico\\
\end{tabular}

\bigskip
\section*{\textsc{Herramientas Computacionales}}

\begin{tabular}{ll}
Matlab (software computación científica) 					& avanzado\\
Xilinx (diseño y simulación de circuitos digitales)		& avanzado\\
Latex (editor de texto)									& avanzado\\
Word, Excel, PowerPoint									& avanzado\\
LTSpice (simulación de circuitos electrónicos)			& medio\\
Eagle (diseño de PBCs)									& medio\\
Lenguajes de programación								& Python, AMPL\\
Lenguajes de descripción de hardware					& Verilog
\end{tabular}

\bigskip
\section*{\textsc{Lenguajes}}

\begin{tabular}{ll}
Español & lengua nativa\\
Ingles & fluido
\end{tabular}

\bigskip
\section*{\textsc{Referencias}}

\begin{itemize}
	\item Aldo Cipriano Zamorano. PhD de la Unversidad de München en Ingeniería Eléctrica. Profesor Titular y académico de la Pontificia Universidad Católica. Director General de DICTUC. Director alterno proyecto FONDEF que enmarcó la tesis de postgrado. Profesor supervisor durante trabajo de tesis de postgrado entre 2016 y 2018. \href{mailto:aciprian@ing.puc.cl}{\underline{aciprian@ing.puc.cl}}
	\item Felipe Nuñez Retamal. Profesor Asociado de la Escuela de Ingeniería de la Pontificia Universidad Católica de Chile. Director proyecto FONDEF que enmarcó la tesis de postgrado. \href{mailto:fenunez@ing.puc.cl}{\underline{fenunez@ing.puc.cl}}
\end{itemize}

\bigskip
\section*{\textsc{Información Adicional}}
\begin{tabular}{ll}
Rut 			& 18.395.108-4 \\
Fecha de nacimiento & 11 de mayo de 1993\\
Lugar de nacimiento & Santiago, Chile\\
%Estado Civil & : Soltero \\
Dirección & Parque Sur 12815 Casa 23, Lo Barnechea\\
Intereses & Investigación, música (batería y guitarra), viajar y deportes (escalada, fútbol)  \\
Trabajos adicionales & Clases particulares de ciencias básicas y desarrollo de software
\end{tabular}
\end{document}
