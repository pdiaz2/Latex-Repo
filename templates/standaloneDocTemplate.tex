\documentclass{article}

\usepackage{amsmath}    % need for subequations
\usepackage{graphicx}   % need for figures
\usepackage{verbatim}   % useful for program listings
\usepackage{color}      % use if color is used in text
\usepackage{subfigure}  % use for side-by-side figures
\usepackage{hyperref}   % use for hypertext links, including those to external documents and URLs
\usepackage[spanish]{babel}
\usepackage{cancel}
\usepackage[utf8]{inputenc}
\usepackage{fullpage}
\usepackage[section]{placeins} %esto evita que imagenes se salgan de su seccion
\usepackage{cite} %referencias
\usepackage{hyperref}
\usepackage{amsfonts}
\usepackage{listings}
\everymath{\displaystyle}

%\usepackage{epstopdf}
% define tabular espaciado

\newenvironment{mitab}[2]{\vspace{#1}\begin{tabular}{#2}}%
		{\end{tabular}\vspace{5mm}}
% define mi tipo de frames
\newenvironment{myFrame}[1]{
\begin{frame}[allowframebreaks]
\frametitle{#1}
\justifying
}
{\end{frame}}

%\newenvironment{myEqn}{\begin{subequations}\begin{gather}}
%{\end{gather}\end{subequations}}
% define listas

\newcommand{\be}{\begin{enumerate}}
\newcommand{\ee}{\end{enumerate}}
\newcommand{\bi}{\begin{itemize}}
\newcommand{\ei}{\end{itemize}}
\newcommand{\bc}{\begin{center}}
\newcommand{\ec}{\end{center}}

% puntaje pruebas
\newcommand{\pts[1]}{{\color{red} $\to$ \boxed{\textbf{#1 pts.}}}}

% cosas matematicas

\renewcommand{\d}{\mathsf{d}}
\newenvironment{eqn}{\begin{eqnarray*}}{\end{eqnarray*}}

\newcommand{\re}[1]{\mbox{Re\{$#1$\}}}		% la parte real
\newcommand{\im}[1]{\mbox{Im\{$#1$\}}}		% la parte imaginaria
\newcommand{\gauss}{\operatorname{Gauss}}		% gauss
\newcommand{\rect}{\sqcap}			% rect
\newcommand{\pillbox}{\operatorname{circ}}		% circ
\newcommand{\sinc}{\operatorname{sinc}}		% sinc
\newcommand{\jinc}{\operatorname{jinc}}		% jinc
\newcommand{\asinc}{\operatorname{asinc}}		% asinc
\newcommand{\sen}{\operatorname{sen}}		% seno
\newcommand{\senh}{\operatorname{senh}}		% seno
\newcommand{\triang}{\wedge}			% triangulo
\newcommand{\allint}{\ds{\int_{-\infty}^\infty}}	% integral de -infinito a +infinito
\newcommand{\midint}{\int_{0}^\infty}		% integral de cero a +infinito
\newcommand{\allsum}[1]{\sum_{#1=-\infty}^\infty} % idem suma
\newcommand{\hor}{\operatorname{\uparrow\uparrow}}		% horquilla
\newcommand{\ahor}{\operatorname{\uparrow\downarrow}}		% anti horquilla
\newcommand{\sgn}{\operatorname{sgn}}			% signo
\newcommand{\ddx}{\frac{d}{dx}}			% d/dx
\newcommand{\ddt}{\frac{d}{dt}}			% d/dt
\newcommand{\partiald}[2]{\frac{\partial#1}{\partial#2}}		% d/dx (parcial)
\newcommand{\der}[1]{#1^{\prime}}		% ' derivada
\newcommand{\derr}[1]{#1^{\prime\prime}}	% '' doble derivada
\newcommand{\FT}[1]{{\cal F}\left\{#1\right\}}		% FT
\newcommand{\IFT}[1]{{\cal F}^{-1}\left\{#1\right\}}
\newcommand{\FTc}[1]{{\cal F}{_C}\{#1\}}
\newcommand{\FTs}[1]{{\cal F}{_S}\{#1\}}
\newcommand{\periodic}[1]{\tilde{#1}}		% funcion periodica
\newcommand{\llave}[4]{ \left\{ \begin{array}{ll}
			#1 & #2 \\
			#3 & #4
			\end{array}
			\right.}
\newcommand{\da}{\longrightarrow}
\newcommand{\noda}{\longleftarrow}
\newcommand{\matlab}{Matlab}
\newcommand{\beq}{\begin{equation}}
\newcommand{\eeq}{\end{equation}}
\newcommand{\bseq}{\begin{subequations}}
\newcommand{\eseq}{\end{subequations}}

%\newcommand{\blin}{\begin{align}}
%\newcommand{\elin}{\end{align}}
%\newcommand{\bal}{\begin{align*}}
%\newcommand{\eal}{\end{align*}}
\newcommand{\baq}{\begin{eqnarray*}}
\newcommand{\eaq}{\end{eqnarray*}}
\newcommand{\eref}[1]{(\ref{#1})}
\newcommand{\ddelta}{{}^2\delta}
\newcommand{\trans}[1]{{#1}^{\ensuremath{\mathsf{T}}}}   % transpose

%       definicion de funciones step y shah
\newlength{\widthfontline}
\setlength{\widthfontline}{0.12ex}

\newcommand{\step}{\text{\mbox{\rule{0.3ex}{0cm}%
                        \rule{\widthfontline}{1.0ex}%
                        \hspace{-\widthfontline}%
                        \rule[1.0ex]{1.5ex}{\widthfontline}}}}
\newcommand{\shah}{\text{\mbox{\rule{0.3ex}{0cm}%
                        \rule{\widthfontline}{1.5ex}%
                        \rule{0.7ex}{\widthfontline}%
                        \rule{\widthfontline}{1.5ex}%
                        \rule{0.7ex}{\widthfontline}%
                        \rule{\widthfontline}{1.5ex}%
                        \hspace{\widthfontline}}}}
\newcommand{\sqsq}{\text{\mbox{\rule{0.3ex}{0cm}%
                        \rule{\widthfontline}{1.2ex}%
                        \rule[1.1ex]{0.7ex}{\widthfontline}%
                        \rule{\widthfontline}{1.2ex}%
                        \rule{0.7ex}{\widthfontline}%
                        \rule{\widthfontline}{1.2ex}%
                        \hspace{\widthfontline}}}}
%%%%%%%%%%%%%%%%%%%%%%%%%%%%%%%%%%%%%%%%%%%%%%%%%%%%%%%%%%%%%%
%%%%%%%%%%%%%%%%%%%%%%%%%%%%%%%%%%%%%%%%%%%%%%%%%%%%%%%%%%%%%%
%%%%%                  Funciones Propias                   %%
%%%%%%%%%%%%%%%%%%%%%%%%%%%%%%%%%%%%%%%%%%%%%%%%%%%%%%%%%%%%%%

\newcommand{\save}[2]{\newcommand{#1}{#2}} %funciona, pero hay que poner el backslash cuando se usa. Para llamarla hay que hacer \save{\#1}{#2}
\newcommand{\diff}[2]{\frac{d #1}{d #2}} % derivada
\newcommand{\vabs}[1]{\left\lvert #1 \right\rvert}%valor absoluto bonito
\newcommand{\corch}[1]{\left[ #1 \right]}
\newcommand{\squig}[1]{\left\lbrace #1 \right\rbrace}
\newcommand{\ptsis}[1]{\left( #1 \right)}
\newcommand{\overbar}[1]{\mkern 1.5mu\overline{\mkern-1.5mu#1\mkern-1.5mu}\mkern 1.5mu}
\newcommand{\logn}[1]{\text{log}\ptsis{#1}}%logaritmo natural
\newcommand{\ttt}[1]{\texttt{#1}}%formato computador, funciones de MATLAB
\newcommand{\qt}[1]{``{#1}''}%poner entre comillas
\newcommand{\tw}{\textwidth} %shortcut pal textiwdth
\newcommand{\mth}[1]{\( #1\)}
\newcommand{\mthd}[1]{\[ #1\]}
\newcommand{\defeq}{\overset{\small{\bigtriangleup}}{=}}
\newcommand{\bsh}{\textbackslash}
\newcommand{\intEval}[2]{\Big|_{#1}^{#2}} % Evaluate integral at limits
\newcommand{\atan}{\text{atan}}
\newcommand{\acos}{\text{acos}}
\newcommand{\asin}{\text{asin}}
\newcommand{\bs}[1]{\boldsymbol{#1}}
\newcommand{\mhr}[1]{$\text{m}^{#1}/\text{hr}$}
\usepackage{stackengine}
\def\delequal{\mathrel{\ensurestackMath{\stackon[2pt]{=}{\scriptscriptstyle\Delta}}}}
\newcommand{\fig}[5]
{
  \begin{figure}
    \centering
    \resizebox{#3}{#4}{\includegraphics{#1}}
    \caption{#5 \normalsize}
    \label{#2}
  \end{figure}
}

\newcommand{\rojo}[1]
{
{\color{red}#1}
}
%%%%%%%%%%%%%%%%%%%%%%%%%%%%%%%%%%%%%%%%%%%%%%%%%%%%%%%%%%%%%%
%% Manejo UTF8 de listings %%%%%%%%%%%%%%%%%%%%%%%%%%%%%%%%%
\graphicspath{{C:/Users/pablo/Ingenieria/Latex/}{./figuras/}}

\usepackage{float}
 
%\setlength\parindent{0pt} %esto hace la sangria nula

\begin{document}
\thispagestyle{empty}
%\vspace*{-1cm}
%\vspace*{-2cm}

\vspace*{-1cm}
\includegraphics[width=2cm]{logo_PUC.pdf}
\vspace*{-2cm}

\hspace*{2cm}
 \begin{tabular}{l}
  {\ Pontificia Universidad Católica de Chile}\\
  {\ Escuela de Ingeniería}\\
  {\ Departamento de Ingeniería Eléctrica}\\
  {\ Magister en Ciencias de la Ingeniería }\\
  {\  }\\
 \end{tabular}
 \hfill 
\vspace*{-0.2cm}
\begin{center}
{\Large\bf Problema de Optimización}\\
\vspace*{2mm}
{\Large Tesis de Postgrado}\\
%\vspace*{3mm}
{15 de diciembre de 2017}\\
\vspace*{1mm}
{\bf Pablo Diaz - 12634581 }\\
\vspace*{1mm}
\end{center}
\hrule\vspace*{2pt}\hrule
%%%%%%%%%%%%%%%%%%%%%%%%%%%%%%
%%%%%%%%% ENCABEZADO %%%%%%%%%
%%%%%%%%%%%%%%%%%%%%%%%%%%%%%%
%\newpage
\setcounter{page}{1}
%\tableofcontents
%\newpage

\section{Formulación MPC}
\subsection{Caldera}
\begin{description}
\item[Variables Controladas] Las variables controladas son presión de vapor, nivel de oxígeno y nivel de agua.
\item[Variables Manipuladas] Las variables manipuladas son combustible, flujo de aire y flujo de agua.
\item[Perturbaciones medidas] La perturbación medida es la demanda de vapor.
\item[Perturbaciones no medidas] A los 90 minutos de operación, se produce un cambio en las propiedades del combustible.

\end{description}
El control de la caldera es un problema con las siguientes dimensiones:

\bseq{}\label{eq:calderaDims}
\begin{gather}
y \in \corch{0,100} \times \corch{0,100} \times \corch{0,100}\\
u \in \corch{0,100} \times  \corch{0,100} \times \corch{0,100}\\
p \in \corch{0,100}\\
\dot{x}(t) = f(x,u), x(0) = x_0
\end{gather}
\eseq{}
Dependiendo del método de identificación de sistemas, se hará uso o no del modelo en espacio de estados (para N4SID se utiliza este modelo). El modelo es no lineal; sin embargo, para efectos de planta de prueba interesa solo un punto de operación, caracterizado por:
\bseq{}\label{eq:opPointCaldera}
\begin{gather}
 y_0 = \corch{41.41 ; 28.75 ;38.60}\\
 u_0 = \corch{32.98 ; 46.22 ;25.34}\\
 d_0 = 35.78
\end{gather}
\eseq{}

\subsubsection{Formulación Restricciones}
En el concurso, se establecen las siguientes restricciones:
\bseq{}\label{eq:restriccionesCaldera}
\begin{gather}
\vabs{\Delta u_i(k)} \leq 1 \:\text{pps} \: \forall i\\
 0.95y_{2_0} \leq y_2 \leq 1.05y_{2_0}
\end{gather}
\eseq{}

Estas restricciones representan un conjunto convexo y lineal de restricciones.

\subsubsection{Formulación Modelo Entrada-Salida}
En el caso del uso de \textit{random forests}, no se utiliza la formulación en espacio de estados. Se utiliza en cambio un modelo entrada-salida \textbf{no lineal y no diferenciable}:

$$y(k+1) = f_{RF}(\squig{u(k-d_u), d_u = {0,\cdots, n_u}  },\squig{y(k-d_y), d_u = {0,\cdots, n_y}},\squig{p(k-d_u), d_p = {0,\cdots, n_p}})$$

En la notación anterior se intenta dejar de forma explícita el uso de valores pasados de las entradas y valores pasados de las salidas. Esto guarda estricta relación con las formulaciones ARMAX y (dependiendo de las propiedades del ruido) CARIMA del control MPC clásico.

La metodología para la obtención del modelo predictivo en base a \textit{random forests} es la siguiente:
\begin{enumerate}
\item Generación de excitaciones sobre la planta de prueba de distinta forma (sinusoidales, cuadradas, escalones y diente de sierra). Estas señales se generan de forma apropiada para que no violen los rangos establecidos para la planta. El proceso se somete a estas excitaciones y combinaciones de ellas simultáneamente, registrándose las salidas.
\item Recopilación de datos entrada-salida.
\item Especificación del llamado \qt{tiempo efectivo de reacción} $\tau_r$. Este tiempo se puede entender análogamente a una constante de tiempo en sistemas lineales: pasado un tiempo efectivo de reacción se observa un cambio razonable en las variables. El tiempo efectivo de reacción se eligió como 5 veces el tiempo de muestreo del sistema (0.5 segundos).
\item Alimentación a algoritmo de \textit{random forests} de las series de tiempo de las señales y combinaciones de estas retrasadas (múltiplos de $\tau_r$). Las combinaciones son elegidas de forma representativa y arbitraria. Este algoritmo se hace con hiper-parámetros estandarizados de 100 árboles y tamaño mínimo de hoja de 10 muestras.
\item Cálculo de métricas relevantes de error: MSE, coeficiente de correlación y \textit{out of bag error}.
\item Selección en base a mejor criterio de métrica escogida (MSE) de los mejores $\kappa_y$ modelos para cada variable $y$.
\item Optimización bayesiana de los hiper-parámetros de los $\kappa_y$ modelos escogidos.
\item Cálculo de mismas métricas de error y selección del mejor modelo $f_y$ para cada variable $y$.
\end{enumerate}

El modelo predictivo obtenido en base a \textit{random forests} actualmente se encuentra en un estado sub-óptimo por los siguientes motivos:
\begin{itemize}
\item Horizonte de simulación quizás insuficiente.
\item Falta de optimización bayesiana del primer conjunto de mejores modelos $\kappa_y$.
\end{itemize}

Una vez obtenido el modelo $f_{RF}$ que es un predictor \textbf{a un paso}, se \qt{eleva} su horizonte de predicción a $N_y$ pasos realimentando las salidas predichas $\hat{y}(k+j)$, $j = {1,\cdots,N_y}$. Existe también un horizonte de control $N_u$. Se tienen las siguientes consideraciones:
\begin{itemize}
\item Cuando $j > N_u$, se asume que las variables manipuladas quedan constantes. En otras palabras, $$\Delta u(k+j) = 0 \: \forall j > N_u$$
\item Las $N_y$ predicciones se pueden ver como $N_y$ \textbf{restricciones} que relacionan las entradas con las salidas. Estas restricciones son del tipo no lineal (debido a la función $f_{RF}$) y por lo tanto hacen necesario otros métodos de optimización.
\item Una primera aproximación podría ser basada en los principios de Bellman y programación dinámica: fijar $u(k)$ y optimizar para esta variable en primera instancia. Una vez encontrado el óptimo para esta (y habiendo asumido que la entrada permanece igual en todo el horizonte $N_y$), optimizar para $u(k+1)$ y así sucesivamente. Esta noción de metodología es búsqueda local de soluciones óptimas; sin embargo, existen métodos que lo transforman en búsqueda global.
\item Otra alternativa (basada por ejemplo en algoritmos genéticos y \textit{pattern search}) es la minimización de todo el vector $u(k+j)$ simultáneamente. Esta será la alternativa preferida.
\end{itemize}

\subsubsection{Función Objetivo}
Se especificarán dos funciones objetivos distintas (dos controladores y experiencias distintas):
\begin{description}
\item[Función objetivo cuadrática] La primera formulación será utilizando el MPC para garantizar estabilidad del sistema mediante una función objetivo cuadrática, de la forma:

\bseq{}\label{eq:funcObj1Caldera}
\begin{gather}
min_{u}J(u,y) = \sum_{k = 1}^{N_y}\ptsis{y(t+k)-w(t+k)}^2+\lambda \sum_{k = 0}^{N_u}\ptsis{\Delta u(t+k)}^2
\end{gather}
\eseq{}

%{\color{red}Dudas pendientes}:
%\begin{itemize}
%\item Inclusión de términos $\epsilon_j$ en restricciones de $y(k+j)$ como variables de holgura en restricciones. Penalización de estos en la función de costos. Inspirado en Jain y paper de MPC-RF.
%\item 
%\end{itemize}

\item[Función objetivo lineal] Esta formulación es la final e ideal, pues corresponde a una arquitectura de control en dos niveles (nivel estabilizante regulatorio y nivel optimizante). 
\end{description}
\end{document}